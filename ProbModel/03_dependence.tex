\section{Двумерные случайные величины. Зависимость}

\begin{enumerate}
\item	Переход к двумерным распределениям. Про плотности (найти картинку двумерной плотности). Про изображение плотности линиями уровня, или цветом, либо скаттерплотом из точек. Напомнить, про изображение плотности при равномерном распределении в области (или неравномерном).
\item	Про одномерные (margin) распределения, формула. \textbf{Задание}. Я рисую плотность (нарисовать плотность (заштриховать ромб)) Нужно нарисовать одномерные распределения.
\item	Про условные распределения, формула плотности. Суть – нормированный срез. Условное математическое ожидание – обычное мат.ож. условного распределения (например, с условной плотностью).
\item	Независимость через произведение плотностей и через одинаковость условных вероятностных распределений. \textbf{Задание}: написать, где есть независимость, где ее нет. (по картинкам).
\item	Зависимость как вид условного мат.ож. Регрессия: $y = \textsf{E}(\eta | \xi = x)$. \textbf{Задание}. Пример с параллелепипедом. Нарисовать линию регрессии.
\item	Виды регрессии – линейная, нелинейная, полиномиальная, …. Меры зависимости. Корреляция, корреляционное отношение. Линейная регрессия и просто регрессия. Свойства корреляции, некоррелированности и независимость, нарисовать картинки.
\item	Коэффициент корреляции, через МНК.
2.	Ковар. и корреляц.матрица. Изображение цветом. Задание. Изобразите корреляц.матрицу для признаков (вес, рост, возраст) для распределения характеристик взрослого человека 30-40 лет и для распределения всех живущих в большом доме в спальном районе.
3.	Многомерное нормальное распределение. Расстояние Махаланобиса (=линии уровня плотности). Линейная регрессия. Формула плотности в невырожденном случае. Двумерный случай, как выглядит плотность распределения в зависимости от мат.ож, дисперсий и корреляции. Стандартное нормальное распределение. Показываю, как связаны параметры и линии уровня. (Рисую три графика с линиями уровня.)  Варианты для корреляции 0.9, 0.4, 0, -0.4, -0.9, сопоставляю корреляц.матрицу, вектор дисперсий и вектор мат.ож.


\item	Ковар. и корреляц.матрица. Изображение цветом. \textbf{Задание}. Изобразите корреляционные матрицы для признаков (вес, рост, возраст) для распределения характеристик взрослого человека и детей.
    Или:  Изобразите корреляц.матрицу для признаков (вес, рост, возраст) для распределения характеристик взрослого человека 30-40 лет и для распределения всех живущих в большом доме в спальном районе.
\item	Многомерное нормальное распределение. Расстояние Махаланобиса (=линии уровня плотности). Линейная регрессия. Формула плотности в невырожденном случае. Двумерный случай, как выглядит плотность распределения в зависимости от мат.ож, дисперсий и корреляции. Стандартное нормальное распределение.
    Показываю, как связаны параметры и линии уровня. (Рисую три графика с линиями уровня.)  Варианты для корреляции 0.9, 0.4, 0, -0.4, -0.9, сопоставляю корреляц.матрицу, вектор дисперсий и вектор мат.ож.

    \textbf{Задание}: нарисовать линиями уровня плотность при мат.ож (1,2), стандартных отклонениях (2,1), корреляции - 0.8.
\end{enumerate}
