\section{Построение оценок}

\begin{enumerate}
\item	Методы построения оценок. Метод подстановки. Метод моментов. \textbf{Задание}: равномерное распределение, Пуассоновское распределение, экспоненциальное распределение.
\item	Определение функции правдоподобия. Максимальное правдоподобие – что делать, если нет учителя. Ответ: максимизировать правдоподобие.
\item	Пример: распределение Бернулли, нормальное распределение при известной дисперсии. Пример: пуассоновское распределение, экспоненциальное распределение. \textbf{Задание}. Найти ОМП. Совпадают ли с оценкой по ММ?
Пример р.р. на $[0,\theta]$, где разные.
\item	\textbf{Задание}. Есть оценка $\theta$, несмещенная. Утв. существует константа c, такая что дисперсия больше c. (или меньше c). Выберите то, что считаете верным.
Неравенство Рао-Крамера.
\item	\textbf{Задача}: есть модель: $a+(laplace(\lambda)), a+N(0, \sigma^2)$. Какие наилучшие оценки для $a$?
\item	Выборка не повторная, разные дисперсии. \textbf{Задание}. Найти оценку ОМП.
\item	Использование ОМП: хи-квадрат, … (вряд ли что-то еще), model-based кластеризация, тематическое моделирование, классификация с помощью логистической регрессии, информационные критерии для выбора модели.
\item	Информационные критерии. $AIC = 2k -2 \ln L$, $BIC = k \ln n - 2 \ln L$.
\textbf{Задание}. Вам нужно сравнить две модели данных, экспоненциальную и логнормальную. Объем выборки 55 (ln 55 примерно равен 4). Exp: подставили ОМП в L и получили exp(-13). Lognorm: exp(-10), т.е. правдоподобие в случае логнормального больше.
\item	Нарисовать картинки и посчитать AIC через SSE.
\end{enumerate}

\section{Робастность}
\begin{enumerate}
\item	Что такое выброс. Выброс по отношению к закономерности. Выброс и неоднородность. Задание. Нарисовать картинки и спросить, где выбросы и по отношению к чему?
\item	Робастность. Выб.ср. и выб.медиана. T-test и MW.
\item	Коэф.корр.Пирсона и Спирмена. \textbf{Задание}: какой коэффициент корреляции больше по модулю?
\item	M-оценки.
\end{enumerate}

\section{Множественное тестирование}
\begin{enumerate}
\item	Множественное тестирование. Проблема: ошибка FWER. Задача: посчитать FWER для независимых тестов. Комикс. \textbf{Задание}: объяснить комикс.
\item	Решение: изменить уровень значимости для одного теста. \textbf{Задание}: изменить на поправку p-value. Этот тест точный, т.е. FWER = alpha
\item	А если тесты не независимые? Тогда поправка Бонферрони.
\textbf{Задание}. В каком случае поправка Бонферрони приведет к максимально консервативному тесту? Если все тесты полностью зависимы (выдают одинаковые p-values).
\item	Иногда получается построить точный тест для зависимых сравнений. Post-hoc comparisons. Таблица с \textbf{заданием}.
\item	(распечатать 6 и 7 из файла)
\end{enumerate}

