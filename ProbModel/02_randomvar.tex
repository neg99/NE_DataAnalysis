\section{Случайные величины}

\begin{enumerate}
\item Случайные величины. Их распределения. \\
Случайная величина $\xi:(\Omega,\mathcal{F},P) \mapsto (\mathbb{R},\mathcal{B})$ --- измеримое отображение. Задает распределение $(\mathbb{R},\mathcal{B},\mathcal{P}_\xi)$. Поэтому говорят о $\mathcal{P}_\xi(A) =  P(\xi\in A)$ и можно рассматривать дискретные распределения (дискретные случайные величины), непрерывные распределения (непрерывные случайные величины), имеющие плотность $p(x) = p_\xi(x)$.
\item \textbf{Задание}. Случайная величина –-- время из (вашего) дома до мат-меха. Нужно нарисовать плотность распределения случайной величины и отметить на оси $x$ значение времени, за которое вы будете выходить из дома 1) на лекцию, 2) на контрольную работу.
\item Функция распределения. Определение $F_\xi(x) = P(\xi < x)$ (но более стандартно $F_\xi(x) = P(\xi \le x) $, разница только для дискретных распределений). Рисунки функций распределения.
Для дискретного распределения, разница в том, < или <=. \textbf{Задание}. Функция распределения случайной величины, которая всегда равна 1?
\item
    Характеристики положения (мат.ож., медиана), характеристики разброса (дисперсия, среднее абсолютное отклонение). Свойства мат.ож. и дисперсии.\\
//Свойства дисперсии. Минимум достигается на мат.ож.
\textbf{Задание}: нарисовать две плотности и соответствующие им функции распределения, с отличием только в среднем или с отличием только в разбросе.
\item Асимметрия и эксцесс. Соотношение мат.ож. и медианы.\\
//Задание: пусть коэффициенты равны …. Чему они равны для $2\xi + 5$?
\textbf{Задание}: Логнормальное распределение зарплаты (плотность нарисована мной). Вам предлагают в качестве начальной зарплаты среднюю зарплату или медианную. Какую вы выберете?
\item	Примеры распределений: Нормальное, правило $k$ сигм. 
Свойства хорошо известны. В частности, плотность имеет вид
\[
p(x) = \frac{1}{\sigma\sqrt{2\pi}}\; e^{ -\frac{(x-a)^2}{2\sigma^2} },
\]
математическое ожидание равно $a$, дисперсия $\sigma^2$, асимметрия и эксцесс равны 0.

Рассмотрим вопрос про измерение расстояния в сигмах.
Будет говорить, что точка далеко от мат.ожидания, если это и более далекие значения маловероятны.

Формально, пусть $\xi\sim N(a,\sigma^2)$. Рассмотрим $\P(|\xi - a|> k \sigma)$. Эта вероятность не зависит от $\sigma$ и равна $2(1- \Phi(k))$, где $\Phi(x)$ --- функция стандартного нормального распределения $\N(0,1)$.

Значения $\P(|\xi - a|> k\sigma)$:\\
\begin{tabular}{l|r}
\hline
$k$& Вероятность\\
\hline
1&	0.317\\
1.64&	0.101\\
1.96&	0.050\\
2	&0.046\\
3	&2.70E-03\\
6	&1.97E-09\\
\hline
\end{tabular}

Отсюда правило двух сигм (вероятность быть на расстоянии от мат.ож. больше двух cигм примерно равна 0.05), правило трех сигм, правило шести сигм.
\textbf{Задание}. Нарисуйте плотность распределения $N(1,4)$.\\
Рост человека может иметь нормальное распределение?
\item
Экспоненциальное (что происходит с мат.ож, когда lambda растет?), $p(x)=\lambda e^{-\lambda x}$, если $x\ge 0$; 0 иначе. $F(x) = 1-e^{-\lambda x}$, $x\ge 0$ (0 иначе)\\
пуассоновское $P(\xi = k) = \frac{\lambda^k}{k!} e^{-\lambda}$ ($k=0,1,2,\ldots,\infty$), Бернулли, биномиальное как сумма Бернулли, геометрическое и пр.
\textbf{Задание}: какое мат.ож. и какая дисперсия у биномиального распределения (через мат.ож. и дисперсию суммы)?
\textbf{Задание}: Одна из формул правильная для мат.ож. геометрического распределения ($p/q$ или $q/p$). Какая (по смыслу) и почему? Считать ничего не нужно.
\item	Сходимость в разном смысле. Очень важно, потому что часто мы что-то знаем про предел и можем этим пользоваться. Например, $S_n$ (накопленные суммы) или среднее арифметическое. Или последовательность случайных величин, которая стремится к константе.
\item	Пример: сходимость по вероятности к константе – начиная с некоторого момента вероятность попасть в eps-окрестность константы стремится к 1.
\item	Сходимость по распределению: функция распределения стремится к предельной ф.р. кроме точек разрыва предельного распределения.
\textbf{Задание} $\xi_n$, $\textsf{E} \xi = 1$, $\textsf{D}\xi = 1/n$. Докажите с помощью картинок, что есть сходимость и по вероятности, и по распределению к 0.
\item	Закон больших чисел. На его основе считаем среднее арифметическое для оценки мат.ож.  Неравенство Чебышёва. Из него следует ЗБЧ.
\item	Центральная предельная теорема. На ней основано огромное количество теоретических результатов. Также, она объясняет, почему нормальная модель хорошо описывает большое количество явлений.
\item	Сходимость для биномиального распределения к нормальному. \textbf{Задание}. Чему примерно равна вероятность того, что при 400 попытках число успехов больше 220.
\item	Если останется время, то про мат.ож. и дисп., про медиану и абс.откл., про виды случ.величин (порядковые, качественные) и их характеристики.
\end{enumerate}
