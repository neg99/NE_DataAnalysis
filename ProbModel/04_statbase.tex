\section{Базовая статистика}

\begin{enumerate}
\item Нас интересует вероятностное распределение (распределение случайной величины). Выборка, выборка одномерная, выборка двумерная, как реализации скаттерплот. \textbf{Задание}: нарисовать выборку согласно нормальному распределению с корр. $-0.4$, средними $1$ и $2$, стандартными отклонениями $2$ и $1$.
\item	Теории на числах не построить. Нужно теперь связать выборку с закономерностью. Поэтому в мат.стат. выборка воспринимается в абстрактном смысле как n независимых случайных величин (и в двумерном случае). \textbf{Задание}: решите задание из теста:  $\textsf{E}(x_1+x_3)$ и $\textsf{E}(x_1 x_2)$ = ?
\item	Эмпирическое распределение. Связываем выборку и случайную величину через эмпирическое распределение. (одномерный и двумерный случай.)
Функция распределения. Гистограмма как эмпирич. распределение.
\item	Таким образом, схема такая:
1) нас интересует характеристика $\xi$\\
2) подставляя вместо неизвестной $\xi$ известную эмпирическую случайную величину, получаем оценку\\
3) рассматриваем выборку как абстрактную (это случайный вектор), получается оценка – случайная величина. Выясняем, насколько оценка хорошая.
\item	Пример 1) и 2) для мат.ож., выборочное среднее. 
\item	Примеры со значением параметра и плотностями оценок. (смещ., не смещ., разная дисперсия). \textbf{Задание}: упорядочить оценки по качеству, с объяснением.
\item	Объяснением: сравниваем по MSE. Получаем сумма дисп. и смещения в квадрате.
\item	Итак, свойства выб.среднего как оценки мат.ож. Нулевое смещение и поэтому дисперсия. Свойства оценок – несмещенность, состоятельность, …  \textbf{Задание}: Посчитайте дисперсию.
\item	Выписываю то же самое для выборочной дисперсии.
\item	Переходим к корреляции. Все то же самое. (на этом закончили)
\item	Переходим к регрессии (МНК). Напоминаю для случ. величин. То же самое для выборки.
\item	Проблемы с выбросами и нелинейными зависимостями.
\item	Пример с ирисами
\item	Надо бы пример с деньгами и логарифмированием.
\end{enumerate}
