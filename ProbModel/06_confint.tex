\section{Доверительные иннтервалы}

\begin{enumerate}
\item	Про доверительные интервалы, в целом. Пример, как построить доверит. интервал для матем.ожидания, без модели и с моделью (что все равно, если нет нормальности). Про распределение Стьюдента, коротко. Исправленная дисперсия.
\item Про асимптотические доверит.интервалы для мат.ож.
\item	Про использование доверит.интервалов для проверки гипотез. Пример с числами
\item	\textbf{Задание}. Сделать выводы (по файлу, приготовить и выложить перед занятием).
\item	Про p в Бернулли. \textbf{Задание}: написать, как будет выглядеть дов.интервал. Рассказать про более точный интервал Wilson’a.
\item	\textbf{Задание}: Построить доверит.интервал для p, если $\bar{x} = 0.5$, $n = 100$ (0) $\gamma = 0.9$, (1) $\gamma = 0.95$ (2) $\gamma = 0.99$. Проверить гипотезу с соотв. альфой, что (0) $p_0 = 0.6$ (1) $p_0=0.4$ (2) $p_0=0.7$.
\item	Интересные примеры.
\item	Проверка гипотезы про нулевую вероятность, крит.область справа.
\item	Про одинаковые числа подряд.
\item	$\gamma = 0.2$, $c_\gamma = 0.25$. $0.2^5 = 0.0003$, $0.2^{10} = 10^{-7}$
\item	$\gamma = 0.7$, $c_\gamma = 1$. $0.7^5 = 0.17$, $0.7^{10} = 0.03$.
\end{enumerate}
